\selectlanguage{english}%

\chapter{Fundamentos Fuzzy}
A lógica fuzzy, ou difusa, foi introduzida originalmente por Zadeh, em seu artigo "Fuzzy Sets" \cite{zadeh}. Sua teoria de conjunto diverge da booleana 
no tratamento dos valores lógicos das variáveis, podendo assumir qualquer valor entre 0 e 1. 

\section{Conjuntos Fuzzy}
\indent De acordo com a teoria de conjuntos clássica, um elemento $x$ qualquer, pode pertencer ou não à um conjunto universo de discurso $U$, $x \in U$. Chamando de $f_u(x)$ a função de pertinência de $x$ ao conjunto $U$. Desta forma, tem-se:

\begin{align}
	f_u(x) : U \rightarrow {0,1}
	&& f_u(x) =
	\begin{cases*}
		1 & se e somente se $x \in U$ \\
		0 & caso contrário
	\end{cases*}
	\label{eqFPertinencia}
\end{align}

Essa definição binária se encaixa bem em problemas restritos, cujo caráter dos sistemas reflita essa separação de estados, por exemplo a paridade ou não de uma das somas dos bits de uma mensagem binária. No entanto, grande parte dos sistemas estudados nas teorias de controle trabalha com grandezas que possuem limites não tão claros assim, como exemplo a temperatura. Apesar de ser matematicamente bem definida, existem descrições como "frio" e "quente" que não podem ser representadas com este conjunto binário, uma vez que são conceitos vagos e imprecisos. A abordagem fuzzy é capaz de tratar a pertinência nestes casos, de onde vem a origem de seu nome "difusa".

\section{Funções de Pertinência}
	\subsection{Variáveis Linguísticas}
	
\section{Inferência}

\subsection{Fuzzyficacão}

\subsection{Regras}
	\subsection{Defuzzyficacão}


\section{Modelo Fuzzy Takagi-Sugeno}


\selectlanguage{brazil}%

