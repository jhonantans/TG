\selectlanguage{english}%
%PORTUGUESE
\resumo{Resumo}
{
	\indent Plantas descritas matematicamente por sistemas não-lineares apresentam desafios para identificação de sistemas e para aplicação de técnicas de controle convencionais. O procedimento mais simples nestas situações é aproximá-los por um modelo local, linearizado, e assumir que essa dinâmica vale para uma região suficientemente grande de operação do processo. Sabe-se que esta abordagem fornece resultados que divergem dos reais à medida que o estado do sistema se afasta do ponto de referência. Abordagens fuzzy, como os modelos propostos por Takagi-Sugeno, são alternativas eficazes para solução deste problema, uma vez que fazem uma interpolação de vários modelos locais.
	
	Este trabalho faz uso da lógica fuzzy, utilizando modelos Takagi-Sugeno, para desenvolver controladores para uma planta de quatro-tanques, um sistema não-linear com acoplamento entre suas variáveis. O objetivo final é implementar este controlador em um CLP industrial e observar seu desempenho.
}
%ENGLISH
\vspace*{2cm}
\resumo{Abstract}
{
	\indent Plants mathematically described by non-linear systems present challenges for the identification of systems and for the application of conventional control techniques. The simplest procedure in these situations is to approximate them by a local, linearized model, and assume that this dynamic holds for a sufficiently large region of process operation. It is known that this approach provides results that diverge from the real ones as the state of the system moves away from the reference point. Fuzzy approaches, such as the models proposed by Takagi-Sugeno, are effective alternatives to solve this problem, since they interpolate several local models.
	
	This work makes use of fuzzy logic, using Takagi-Sugeno models, to develop controllers for a four-tank plant, a nonlinear system with degrees of coupling between its variables. The final goal is to implement this controller in an industrial PLC and observe its performance.
}\selectlanguage{brazil}%

