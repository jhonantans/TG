\selectlanguage{english}%
%PORTUGUESE
\resumo{Resumo}
{
	\indent Plantas descritas matematicamente por sistemas não-lineares apresentam desafios para modelagem de sistemas e para aplicação de técnicas de controle convencionais. O procedimento mais simples nestas situações é aproximá-los a um estado local, linearizado, e assumir este comportamento pontual como o global do sistema. Sabe-se que esta abordagem fornece resultados que se afastam dos reais à medida que o estado do sistema distoa daquele ponto de referência. Abordagens fuzzy, como os modelos propostos por Takagi-Sugeno, são alternativas geralmente mais eficazes para solução deste problema, uma vez que fazem uma interpolação de várias modelagens em diversos pontos locais.
	
	Este trabalho faz uso da lógica fuzzy, utilizando modelos Takagi-Sugeno, para desenvolver controladores para uma planta de quatro-tanques, um sistema não-linear com acoplamento entre suas variáveis. O objetivo final é implementar este controlador em um CLP industrial e observar seu desempenho.
}
%ENGLISH
\vspace*{2cm}
\resumo{Abstract}
{
	\indent Plants mathematically described by non-linear systems present challenges for system modeling and for applying conventional control techniques. The simplest procedure in these situations is to approximate these systems to a linear local state and assume this punctual behavior as a global system. It is known that this approach yields results that deviate from the real ones as the state of the system moves away from that reference point. Fuzzy approaches, such as the models proposed by Takagi-Sugeno, are generally more effective alternatives to solve this problem, since they interpolate several models in several local points.
	
	This work makes use of fuzzy logic, using Takagi-Sugeno models, to develop controllers for a four-tank plant, a nonlinear system with degrees of coupling between its variables. The final goal is to implement this controller in an industrial PLC and observe its performance.
}\selectlanguage{brazil}%

