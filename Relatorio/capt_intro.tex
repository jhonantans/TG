\selectlanguage{english}%

\chapter{Introdução} \label{capIntrod}
Desenvolver controladores para sistemas não-lineares é quase sempre uma tarefa dispendiosa e complexa. Para plantas multivariáveis este desafio é ainda maior. Por este motivo que é prática comum recorrer-se à linearização das equações que as descrevem, o que fornece uma aproximação do sistema inicial num formato que se encaixa às teorias de controle convencionais.

A linearização simples, realizada por meio da série de Taylor, resulta uma aproximação excelente localmente. No entanto, à medida que as variáveis controladas e manipuladas se afastam do ponto de operação, condição na qual foi realizada a linearização, o modelo passa a se afastar da planta real.

Neste cenário, a abordagem fuzzy figura como excelente ferramenta para solução destes desvios. Aparecendo pela primeira vez em \cite{zadeh}, foi aplicada à modelagem de sistemas em \cite{takagiSugeno}. Seus métodos consistem na linearização simples do sistema em mais de um ponto, baseados em um conjunto de métricas relevantes para o problema em questão. Desenvolve-se então um conjunto de regras para determinar o grau de pertinência do estado do sistema à cada um dos pontos pré-modelados. Utiliza-se então como modelo a soma ponderada das linearizações por estes coeficientes de pertinência. 

O objeto de estudo deste trabalho será o sistema de quatro tanques, desenvolvido por Karl Johansson \cite{johansson2} com o objetivo didático de demonstrar de forma ilustrativa conceitos e propriedades de sistemas com múltiplas entradas e saídas (MIMO, do inglês \textit{multiple input, multiple output}). Ele consiste em quatro tanques interconectados, um reservatório inferior, quatro válvulas esferas e duas bombas de corrente contínua que bombeiam o fluido do reservatório inferior para os tanques de forma cruzada, de acordo com a razão entre os fluxos definida pela posição das válvulas.

O sistema de quatro tanques é não linear. Seu modelo linearizado apresenta um zero multivariável que pode estar localizado tanto no semi-plano esquerdo quanto no semi-plano direito dependendo da configuração das válvulas. A abertura delas determina se o sistema é de fase mínima ou de fase não-mínima afetando a estratégia de controle a ser adotada.

O objetivo é controlar os níveis do fluido nos tanques inferiores 1 e 2. As entradas do processo são as tensões de entrada das bombas, e as saídas são os estes níveis. As demais variáveis de processo são os níveis do fluido nos tanques 3 e 4, os fluxo da bomba e a razão entre os fluxos para os tanques. 

\section{Organização do Trabalho}
Este trabalho está organizado da seguinte fomra: No \jhhref{capDescSis}{capítulo} são apresentados a planta estudada e o CLP Rockwell onde os algoritmos são implementados. Em seguida, o \jhhref{capFundFuzzy}{capítulo} apresenta uma introdução aos conceitos da lógica e modelagem fuzzy e como aplicá-los. No \jhhref{capMod}{capítulo} são realizadas as três formas de modelagem do sistema abordadas neste trabalho: não-linear, linear e o modelo Takagi-Sugeno. No \jhhref{capControle}{capítulo} o projeto do controlador é desenvolvido, seguindo os conceitos de estabilidade baseados em desigualdade lineares matriciais. O \jhhref{capImp}{capítulo} apresenta a implementação dos algoritmos no CLP, utilizando as linguagens de programação aceitas por este. No \jhhref{capRes}{capítulo} são apresentados os resultados das simulações e do sistema real. Por fim as considerações finais são apresentadas no \jhhref{capConclusao}{capítulo}.


\selectlanguage{brazil}%

