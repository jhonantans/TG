\selectlanguage{english}%

\chapter{Introdução} \label{capIntrod}
\epigraph{Não há assunto tão velho que não possa ser dito algo de novo sobre ele.}{Fiodor Dostoievski}

Desenvolver controladores para sistemas não-lineares é quase sempre uma tarefa dispendiosa e complexa. Para plantas multivariáveis não-lineares este desafio é ainda maior. É por este motivo que é prática comum recorrer-se à linearização das equações que as descrevem, obtendo uma aproximação do sistema inicial num formato que se encaixa às teorias de controle convencionais para sistemas lineares.

Esta linearização simples, realizada por meio da série de Taylor, resulta numa aproximação local do sistema não-linear. No entanto, à medida que as variáveis controladas e manipuladas se afastam deste ponto de operação, condição na qual foi realizada a linearização, o modelo passa a se afastar da planta real.

Neste cenário, a abordagem fuzzy figura como excelente ferramenta para solução destes desvios. Aparecendo pela primeira vez nos trabalhos de Zadeh \cite{zadeh}, foi desenvolvida para aplicações em modelagem de sistemas nos trabalhos de Takagi e Sugeno \cite{takagiSugeno}. Seus métodos consistem na linearização convencional do sistema em múltiplos pontos escolhidos criteriosamente, baseados em um conjunto de métricas relevantes para o problema em questão. A partir daí desenvolve-se um conjunto de regras para determinar o grau de conformidade de cada estado do sistema à cada um dos pontos pré-modelados. Utiliza-se então como modelo a soma (uma interpolação) dos múltiplos modelos iniciais ponderada por estes coeficiente de pertinência. 

O objeto de estudo deste trabalho é o sistema de quatro tanques, desenvolvido por Karl Johansson \cite{johansson2} com o objetivo didático de demonstrar de forma ilustrativa conceitos e propriedades de sistemas com múltiplas entradas e saídas (MIMO, do inglês \textit{Multiple Input, Multiple Output}). Ele consiste em quatro tanques interconectados, um reservatório inferior, duas válvulas esferas e duas bombas de corrente contínua que bombeiam o fluido do reservatório inferior para os tanques de forma cruzada, de acordo com a razão entre os fluxos definida pela posição das válvulas. O sistema de quatro tanques é não linear. Seu modelo linearizado apresenta um zero multivariável que pode estar localizado tanto no semi-plano esquerdo quanto no  direito dependendo da configuração das válvulas. A abertura delas determina se o sistema é de fase mínima ou de fase não-mínima afetando a dinâmica geral entre entradas e saídas.

\section{Objetivos}
O objetivo geral é desenvolver um controlador fuzzy, baseado no modelo Takagi-Sugeno da planta, capaz de controlar os níveis do fluido nos tanques inferiores 1 e 2. As variáveis manipuladas do processo são somente as tensões de entrada das bombas, que influenciam de maneira proporcionalmente direta no fluxo.

Como objetivos específicos apresentam-se a instalação do CLP e sua integração. Segue-se então a modelagem e linearização da planta de quatro tanques. Em seguida, a identificação do modelo Takagi-Sugeno proposto juntamente com as funções de pertinência a serem utilizadas, das regras e dos modelos que as compõem. Após a simulação e validação, propõe-se o desenvolvimento de ganhos que estabilizem o sistema em malha fechada, seguindo as referências desejadas. Por fim, objetiva-se a identificação da planta real em 4 pontos a serem utilizados para determinação de seu modelo TS e de ganhos. Finaliza-se com sua implementação e observação dos resultados.

\section{Organização do Trabalho}
Os capítulos iniciais deste trabalham tratam da teoria fuzzy e sua aplicação em sistemas controlados. Já os capítulos finais aplicam essa teoria diretamente sobre a bancada de quatro-tanques e por meio de LMIs são desenvolvidos controladores para ela. No \jhhref{capDescSis}{Capítulo} são apresentados a planta estudada e o CLP Rockwell onde os algoritmos são implementados. Em seguida, o \jhhref{capFundFuzzy}{Capítulo} apresenta uma introdução aos conceitos da lógica e modelagem fuzzy e como aplicá-los. No \jhhref{capMod}{Capítulo} são realizadas as três formas de modelagem do sistema abordadas neste trabalho: não-linear, linear e o modelo Takagi-Sugeno. No \jhhref{capControle}{Capítulo} o projeto do controlador é desenvolvido, seguindo os conceitos de estabilidade baseados em desigualdade lineares matriciais. O \jhhref{capImp}{Capítulo} apresenta a implementação dos algoritmos no CLP, utilizando as linguagens de programação aceitas por este. No \jhhref{capRes}{Capítulo} são apresentados os resultados das simulações e do sistema real. Por fim as considerações finais são apresentadas no \jhhref{capConclusao}{Capítulo}.

\selectlanguage{brazil}%

