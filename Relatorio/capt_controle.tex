\selectlanguage{english}%

\chapter{Controle Fuzzy} \label{capControle}
A sintonia de controladores para os modelos fuzzy deve garantir estabilidade no sistema final, formado pelo conjunto convexo dos pontos do modelo Takagi-Sugeno. Neste trabalho segue-se a mesma metodologia aplicada por Mozelli \cite{mozelli}, utilizando LMIS (\textit{Desigualdades Lineares Matriciais}), baseadas no método de Lyapunov \cite{lyapunov}, para a sintonia dos ganhos utilizados em malha fechada.

\section{Método de Lyapunov}
O método direto de Lyapunov é baseado na positividade de funções. Segue um embasamento para esta última:

\begin{mydef}
Uma função escalar contínua $w: \mathbb{R}^n \rightarrow \mathbb{R}$, $w(0) = 0$ é semidefinida positiva se, e somente se

	\begin{equation}
		w(x) \geq 0, \forall \ \ x \in \mathbb{R}^n - {0}
	\end{equation}
	Caso desigualdade seja estrita, então w será definida positiva. Uma função g é dita semidefinida (definida) negativa caso -g seja semidifenida (definida) positiva
\end{mydef}

O método de Lyapunov baseia-se teorema a seguir:

\begin{myteo} \label{teoLyapunov}
	Um sistema dinâmico autônomo é globalmente estável se existe uma função escalar $V : \mathbb{R}^n \rightarrow \mathbb{R} $ tal que:
	
	\begin{itemize}
		\item V é definida positiva
		\item V possui derivada de primeira ordem
		\item $\dot{V}$ é definida negativa
		\item $V \rightarrow \infty$ a medida em que $\|x\| \rightarrow \infty$
	\end{itemize}
\end{myteo}

Assim, uma função V que satisfaça todo os requisitos do Teorema 1 para um dado sistema é chamada de função de Lyapunov. 


\section{Estabilidade Fuzzy}
Para um modelo TS discreto, simular ao na \jhhref{eqModTakSug}{equação}, tem-se:
\begin{align*}
	x_{k+1} = \frac{\sum_{i=1}^{r}  w_i(c_k)(A_i x_k)}{\sum_{i=1}^{4} w_i(c_k)} 
\end{align*}

Simplificando os termos de ativação de forma a obter um parâmetro A(c), definido como:
\begin{align*}
	h_i[c_k] &:= \frac{w_i[c_k]}{\sum_{j=1}^{r}w_j[c_k]} \\
	A(q) &:= \sum_{i=1}^{r} h_i(c_k)A_i
\end{align*}

Assim,
\begin{align}
	x_{k+1} = A(c) x_k
\end{align}

\begin{equation}
	V(x_k) = x_k' P x_k
\end{equation}
Com P simétrica.

\begin{align}
	\Delta V(x_k) &= x_k'\{A'(c)PA(c) - P\}x_k
\end{align}

De acordo com o \jhhref{teoLyapunov}{Teorema}, se as condições a seguir forem cumpridas, o sistema será assintoticamente estável:
\begin{align}
	P = P' \succ 0 \\
	x_k'\{A'(c)PA(c) - P \}x_k \prec 0 \label{eqLyapXk}
\end{align}
Uma vez que o modelo final é convexo, basta tratar a \jhhref{eqLyapXk}{equação} nos vértices do sistema, ou seja, em cada um dos sistemas que compõem as \jhhref{eqRegraIGeral}{regras}. 
A análise computacional via LMIs possibilita a busca por uma matriz P que satisfaça essas condições, caso ela exista, prova-se a estabilidade do sistema. Assim, o sistema TS é globalmente assintoticamente estável \cite{tanakaWang} existe solução para:

\begin{align}
	encontre \ \ &P \nonumber \\
	s.a \ \ &P \succ 0 \nonumber \\
	&A_i'PA_i - P \prec 0, \ \ i=1,2,3, ... , r
\end{align}

Seguindo estas premissas, o desenvolvimento do controlador para o modelo TS é demonstrado nos trabalho de Wang \cite{wang}, de onde obtém-se o seguinte teorema:
\begin{myteo} \label{teoControlador}
Dado um sistema fuzzy Takagi-Sugeno, convexo, composto por r regras e modelos correspondentes, em malha fechada com ganhos $K_i = M_i X^{-1}$, sua estabilidade é verificada se existe solução para o problema
	\begin{align} \label{eqContFuzzy}
		encontre \ \ &X = X', \ \ i = 1,2,...,r \nonumber \\
		s.a \ \ & 
		\begin{bmatrix}
			X	&	M_j'B_i' + XA_i' \\
			A_iX + B_iM_j &	X
		\end{bmatrix}P \succ 0 \nonumber \\
		& \ \ i,j = 1,2,3, ... , r
	\end{align}
\end{myteo}

Assim, neste trabalho, os modelos fuzzy 

\selectlanguage{brazil}%

